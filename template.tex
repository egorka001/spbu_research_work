%шаблон для НИР
%переработка шаблона для ВКР: github.com/itonik/spbu_diploma/

\documentclass[a4paper, 12pt]{article}
\usepackage{template}

\begin{document}

\begin{titlepage}
    \begin{center}
        САНКТ-ПЕТЕРБУРГСКИЙ ГОСУДАРСТВЕННЫЙ УНИВЕРСИТЕТ \\
        Направление: <<Название направления>> \\ 
        ООП: <<Название программы>> 

        \vspace{6cm}
        {\bfseries ОТЧЁТ О НАУЧНО-ИССЛЕДОВАТЕЛЬСКОЙ РАБОТЕ} 
        \vspace{\baselineskip}
        \begin{flushleft}
            {\bfseries Тема задания:} 
                \begin{minipage}[t]{0.8\textwidth}
                    Исследование влияния пирожков с мясом на качество обучения\\ студентов
                \end{minipage}\\
            \vspace{\baselineskip}
            {\bfseries Выполнил:} Иванов Иван Иванович, группа 19.Б80--ПУ \\
            \vspace{\baselineskip}
            {\bfseries Руководитель НИР:} 
                \begin{minipage}[t]{0.7\textwidth}
                    профессор, кафедра волшебных технолгий и систем,\\  Сидоров И.\:И.
                \end{minipage}
        \end{flushleft}

        \vspace{\fill}
        Санкт-Петербург \\
        1905
    \end{center}
\end{titlepage}
\addtocounter{page}{1}

\newpage
\tableofcontents

\newpage
\specialsection{ВВЕДЕНИЕ}
Есть над чем задуматься: сделанные на базе интернет-аналитики выводы освещают чрезвычайно интересные особенности картины в целом, однако конкретные выводы, 
разумеется, ассоциативно распределены по отраслям. Безусловно, социально-экономическое развитие обеспечивает актуальность системы массового участия. 
Равным образом, реализация намеченных плановых заданий является качественно новой ступенью новых принципов формирования материально-технической и кадровой базы.

Безусловно, внедрение современных методик, в своём классическом представлении, допускает внедрение стандартных подходов. В частности, социально-экономическое 
развитие требует от нас анализа анализа существующих паттернов поведения. Есть над чем задуматься: сторонники тоталитаризма в науке разоблачены!

\newpage
\specialsection{Постановка задачи}
Картельные сговоры не допускают ситуации, при которой многие известные личности объявлены нарушающими общечеловеческие нормы этики и морали. 
Принимая во внимание показатели успешности, социально-экономическое развитие позволяет выполнить важные задания по разработке благоприятных перспектив. 
Являясь всего лишь частью общей картины, сторонники тоталитаризма в науке призывают нас к новым свершениям, которые, в свою очередь, должны 
быть объективно рассмотрены соответствующими инстанциями.

И нет сомнений, что некоторые особенности внутренней политики рассмотрены исключительно в разрезе маркетинговых и финансовых предпосылок. 
Каждый из нас понимает очевидную вещь: сплочённость команды профессионалов является качественно новой ступенью форм воздействия. Принимая во внимание показатели 
успешности, социально-экономическое развитие в значительной степени обусловливает важность позиций, занимаемых участниками в отношении поставленных задач.

\newpage
\section{Как пользователься шаблоном}
Принимая во внимание показатели успешности, постоянное пропагандистское обеспечение нашей деятельности не оставляет шанса для благоприятных перспектив! 
Картельные сговоры не допускают ситуации, при которой активно развивающиеся страны третьего мира набирают популярность среди определенных слоев населения, а значит, 
должны быть подвергнуты целой серии независимых исследований. Также как сложившаяся структура организации однозначно фиксирует необходимость направлений 
прогрессивного развития.

\subsection{Вставка формул}
Формулы вставляются таким образов, главное не забывать про лейбл и ссылки.
\begin{equation}
    a^2 + 2ab + b ^2 = (a + b)^2,
    \label{eq_name_1}
\end{equation}
где $a$ -- количество кроссовок, $b$ -- их качество.

И если слишком долго думать, то в какой-то момент вспомнишь, что у нас есть формула \ref{eq_name_1}, которая всё сразу объясняет.
\begin{equation}
    a^2 + 2ab + b ^2 = (a + b)^2,
    \label{eq_name_2}
\end{equation}
\begin{equation}
    a^2 - 2ab - b^2 = (a - b)^2,
    \label{eq_name_3}
\end{equation}
где $a$ -- количество пирожков с капустой, $b$ -- их качество.

И если слишком долго думать, то в какой-то момент вспомнишь, что у нас есть формулы \ref{eq_name_2} и \ref{eq_name_3}, которые опять всё сразу объясняют.

\subsection{Вставка каритнок}
Опять же картинки вставляются не очень сложно, главное не забывать про лейбл и ссылки. Главное что картинка подписывается правильно.
\begin{figure}[h]
    \center{\includegraphics[width=0.2\linewidth]{img/cat1}}
    \caption{\label{img_cat}Очень милый кот}
\end{figure}

Кстати, сошлёмся на Рисунок \ref{img_cat}, и это очень легко и удобно сделать.
\begin{figure}[h]
    \begin{minipage}[h]{0.5\linewidth}
        \center{\includegraphics[width=0.5\linewidth]{img/cute1}}
    \end{minipage}
    \begin{minipage}[h]{0.5\linewidth}
        \center{\includegraphics[width=0.5\linewidth]{img/cute2}}
    \end{minipage}
    \caption{\label{img_cute}Три милашки переглядываются}
\end{figure}

На Рисунке \ref{img_cute} показано, как можно делать по несколько картинок.

\subsection{Вставка таблиц}
Теперь вставим какую-либо таблицу, и сделаем к ней правильную подпись. Таблицы можно делать сложнее, всё ограничено только фантазией.
\begin{table}[h]
    \begin{center}
        \begin{tabular}{|c|c|c|}
            \hline
            1 & 2 & 3 \\
            \hline
            4 & 5 & 6 \\
            \hline
            7 & 8 & 9 \\
            \hline
        \end{tabular}
    \end{center}
    \caption{\label{keypad}Пример клавиатуры}
\end{table}

Сошлёмся на таблицу \ref{keypad}. Какая интересная таблица, не правда ли?

\subsection{Вставка литературы}
Тут тоже всё легко, просто делай так, и будет ссылка на книгу \cite{book1} и книгу \cite{book2}.
Главно правильно сам список литературы оформить, а это уже привет ГОСТ и прочее.

\section{Текст}
Внезапно, многие известные личности будут объединены в целые кластеры себе подобных. Господа, высокое качество позиционных исследований говорит о возможностях 
глубокомысленных рассуждений. Являясь всего лишь частью общей картины, явные признаки победы институционализации являются только методом политического участия и 
ограничены исключительно образом мышления.

\subsection{Текст подраздела}
Равным образом, сплочённость команды профессионалов позволяет оценить значение дальнейших направлений развития. А также сторонники тоталитаризма в науке 
ассоциативно распределены по отраслям. Противоположная точка зрения подразумевает, что сторонники тоталитаризма в науке являются только методом политического участия 
и преданы социально-демократической анафеме.

\newpage
\specialsection{ЗАКЛЮЧЕНИЕ}
Лишь многие известные личности набирают популярность среди определенных слоев населения, а значит, должны быть указаны как претенденты на роль ключевых факторов. 
Каждый из нас понимает очевидную вещь: семантический разбор внешних противодействий способствует повышению качества глубокомысленных рассуждений! 
Не следует, однако, забывать, что экономическая повестка сегодняшнего дня обеспечивает широкому кругу (специалистов) участие в формировании первоочередных требований.

\newpage
\begin{thebibliography}{1}
    \bibitem{book1} Ландау Л.\:Д., Лифшиц Е.\:М. Теоретическая физика. Т. 1. Механика. М.: Наука, 1988. 216 с.
    \bibitem{book2} Быков В.\:Г. От маятника к роботу. Введение в компьютерное моделирование управляемых механических систем. СПб.: Наука, 2011. 86 с.
\end{thebibliography}

\end{document}
